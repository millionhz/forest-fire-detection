\section{Introduction}

Forest fires have become a major environmental concern across the world, causing significant damage to natural resources, wildlife, and human property. Early detection and quick response are essential to preventing these fires from spreading and causing further destruction. In recent years, computer vision techniques have been used for the early detection of forest fires using visual data. Various models have been proposed for smoke and fire detection, including traditional color-based and deep learning-based models \cite{vision_pedro}. However, accurate and efficient segmentation of the fire region from the background remains a challenging task.

In this paper, we review various traditional models proposed for smoke and fire detection, including color-based methods and CNN models. Several researchers have combined color with textures and dynamic features \cite{detection_zhao} to improve the efficiency of color-based methods. Others have used SVMs \cite{dynamic_ye}, contour lines, and spectral features \cite{an_robust_2022} to improve the accuracy of detection. Deep learning models have also been used for wildfire detection, including models that detect fires both during the day and at night \cite{calderara_vision_2011}.

This research paper presents an approach to semantic segmentation of forest fire images using deep learning models, including DeepLab-ResNet \cite{chen_rethinking_2017}, LRASPP-MobileNet \cite{howard2019searching}, DeepLab-MobileNet \cite{chen_rethinking_2017}, and SegFormer \cite{xie_segformer_2021}. The aim is to accurately identify the fire region and distinguish it from the background, using a range of models with varying complexities and strengths. This allows for an evaluation of the performance of these models on the specific task of forest fire detection.

We describe in detail the different models used in the study, including their architectures, pre-training, and modifications made for the specific task of semantic segmentation of forest fire images. The models used in this study are DeepLab-ResNet, LRASPP-MobileNet, DeepLab-MobileNet, and SegFormer. Each model is evaluated on the task of semantic segmentation, and the results are compared and analyzed.

Overall, this research paper provides a comprehensive evaluation of deep learning models for the task of semantic segmentation of forest fire images. The results highlight the strengths and weaknesses of each model and provide insights into the most effective approach for accurate and efficient forest fire detection. This research can have important implications for the development of automated systems for early detection of forest fires, which can help reduce the risk of wildfires and minimize the damage caused by them. The implementation for this project is available on GitHub \cite{hamza_saqib}, making it accessible for other researchers and developers to build upon and extend the work presented in this paper.
