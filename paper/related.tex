\section{Related Work}

\subsection{Traditional Models}
There are various visual models that have been proposed for smoke and fire detection, mainly including image color methods and CNN models. Deep learning models for wildfire detection \cite{vision_pedro} have made significant progress in recent years. Smoke is an important factor for early detection models and has been the focus of several research projects. Calderara et al. \cite{calderara_vision_2011} used an energy function that extracts smoke color information to detect fires, and created a model that can detect fires both during the day and at night. Yuan \cite{an_robust_2022} used contour lines and spectral features for detection and developed a method with better generalization and insensitivity to shape changes. However, because smoke color is common in everyday objects and makes color-based determination less efficient, some researchers \cite{detection_zhao} combine color with textures and dynamic features to improve performance. Ye et al. \cite{dynamic_ye} performed a multi-scale decomposition of fire images and combined this with the use of SVMs to improve accuracy by simultaneously considering spatial and temporal information of image sequences. Islam et al. \cite{segment_islam} achieved a fire classification accuracy of 97.3\% using smoke color and dynamics properties. However, the model cannot be used for accidental fires far from the camera. Similarly, Han et al. \cite{gaussian_han} combines Gaussian mixture model-based background subtraction with multicolor features to preserve fire motion features and color information.

\subsection{Contemporary Work}
The aforementioned traditional methods are prone to false alarms as they can be affected by other objects resembling fire. To account for false alarms, a DCNN architecture \cite{saliency_zhao} for wildfire detection was proposed and its anti-noise performance was also demonstrated. CNNs have shown good performance in object classification, feature extraction, and target detection. However, CNNs require a large amount of time and storage space for object classification. Muhammad et al. \cite{early_fire_muhammad} used GoogLeNet to create a cost-effective deep learning architecture for fire detection. With only about 5 million parameters, GoogLeNet is much more efficient than VGGNet. To improve dynamic fire detection capabilities, Mao et al. \cite{multichannel_mao} first segmented the flame features and then trained a CNN using a combination of random gradient descent and momentum correction. Saeed et al. \cite{early_fire_saeed} first used Adaboost-MLP to predict fires; then he proposed Adaboost-LBP model and CNN to detect fires. Recently, Ross et al. \cite{hierarchies_girshick} proposed Faster R-CNN: this algorithm uses a region proposal network (RPN) to replace the selective search algorithm and reduce computation time while maintaining accuracy. Faster R-CNN mainly includes convolutional layers, candidate box recommendation networks, feature aggregation layers, and classification layers. This model can realize feature extraction, candidate box extraction, boundary regression and classification integration. But you have to do a lot of operations because you have to classify each candidate box. On the other hand, there is also the problem of alignment between the original image and the feature image, which affects the accuracy of the model. Another type of convolutional neural network is YOLO \cite{yolo_bochkovskiy} (You Only Look Once), which consists of four components: input port, backbone network, neck network, and detection header. This model enhances network feature merging and reduces the problem of losing small target features. This kind of model saves a lot of training time and improves recognition speed. All of these methods are significant improvements over traditional image-based methods.